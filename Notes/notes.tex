\documentclass[12pt]{book}

\usepackage[acronym]{glossaries}
\usepackage{hyperref}
\usepackage{listings}
\usepackage{amsmath}
\usepackage{amssymb}
\usepackage{mathtools}
\usepackage{systeme}
\usepackage{algorithm,algpseudocode}
\usepackage{afterpage}
\usepackage[a4paper, total={6in, 10in}]{geometry}
\usepackage{scrextend}
\usepackage{tcolorbox}
\usepackage{polynom}

\pagestyle{plain}

\newcommand{\example}[1]{
    \begin{addmargin}[1em]{1em}
    \begin{tcolorbox}
    #1
    \end{tcolorbox}
    \end{addmargin}
}

\newcommand{\mychapter}[1]{
    \renewcommand\chaptername{}
    \renewcommand\thechapter{}
    \chapter{#1}
    \renewcommand\thechapter{1}
}

\newcommand{\code}[1]{
    {\color{blue}\texttt{#1}}
}

\definecolor{dkgreen}{rgb}{0,0.6,0}
\definecolor{gray}{rgb}{0.5,0.5,0.5}
\definecolor{mauve}{rgb}{0.58,0,0.82}

% This `if` statement stands here just to comment out 
% examples of figure and listing.
\iffalse
% Example of figure.
\begin{figure}[h]
\centering
\includegraphics[scale=0.5]{Resources/Lecture_2/co_np.png}
\caption{Co-NP problems}
\label{fig:co_np}
\end{figure}

% Example of listing.
\begin{lstlisting}[
    language=Verilog,
    caption=SystemVerilog packed/unpacked arrays,
    columns=flexible,
    basicstyle={\footnotesize\ttfamily},
    numbers=left,
    numberstyle=\tiny\color{gray},
    keywordstyle=\color{blue},
    commentstyle=\color{dkgreen},
    stringstyle=\color{mauve},
    tabsize=4
]
bit [3:0] data; // packed array
logic queue [3:0]; // unpacked array
\end{lstlisting}

% Example of table.
\begin{center}
\begin{tabular}{|c|c|c|}
    \hline
    & this entry & mirror entry \\
    \hline
    distance to the closest cluster & 10 & 6 \\
    \hline
    distance to the second closest cluster & 12 & 8 \\
    \hline
\end{tabular}
\end{center}
\fi



\begin{document}

\mychapter{Linux driver for HC-06 Bluetooth module}
\section{Overview}
This project implements a Linux device driver for HC-06 bluetooth module, which is usually used to give different 
microcontrollers a bluetooth capability (see reference manual for HC-06 bluetooth device: 
\href{https://components101.com/sites/default/files/component_datasheet/HC06%20Bluetooth%20Module%20Datasheet.pdf}{
HC-06 module reference manual}).

The whole \emph{gist} of this project will be based around using this hardware module directly with Linux machine, 
i.e. PC, rather than a microcontroller with UART interface. Thus, for this purpose, we will write a Linux module
for this bluetooth device in order to be able to load it, whenever we need it, and test it on some bluetooth 
compatible devices.

The board that us used in this project, where the actual HC-06 module is placed on is an Arduino compatible board
with description of it being provided here: \href{https://components101.com/wireless/hc-06-bluetooth-module-pinout-datasheet}{
HC-06 board description}. From now on, in the context of this project, this board itself will be called as HC-06, for the 
convenience of it.

In order to communicate with the board, we have to be able to convert USB communication to UART, thus FTDI module 
is used for it: 
\href{https://www.digitec.ch/en/s1/product/deek-robot-dk-ftdi-isp-breakout-335v-power-component-electronics-modules-5997850}{
FTDI board description}.

\section{Pinout and connection}

\section{Interfacing with HC-06 module via command line interface}
We can interface with HC-06 module via passing it commands (including the fact that there is a FTDI module between our machine 
and HC-06 to be able to convert USB communication to UART), as it is mentioned here: 
\href{https://components101.com/wireless/hc-06-bluetooth-module-pinout-datasheet}{HC-06 board description} with actual commands 
being described here: \href{http://wiki.sunfounder.cc/images/7/7b/HC-06_AT_Commands_Reference.pdf}{HC-06 commands}.

\section{Linux device driver}
As it is mentioned in 
\href{https://www.digitec.ch/en/s1/product/deek-robot-dk-ftdi-isp-breakout-335v-power-component-electronics-modules-5997850}{
FTDI board description}, Linux already has a Virtual COM Port (VCP) driver by default loaded for FTDI, which are 
\code{ftdi\_sio} and \code{usbserial} (although, in my Linux installation, which is Arch 6.12.4-arch1-1, \code{usbserial} 
module wasn't loaded and only \code{ftdi\_sio} was loaded). Thus, if we want to write our own driver for FTDI, we would 
have to unload those \code{ftdi\_sio} and \code{usbserial} modules (only \code{ftdi\_sio} in my case, as \code{usbserial}
is not even loaded on my machine).

Once we connect our FTDI module to our machine a device file can be found in \code{/dev/} directory, in my case, it is 
\code{/dev/ttyUSB0}. This way, we can communicate with our FTDI module via writing to and reading from \code{/dev/ttyUSB0}
file.

\section{Communication via terminal}
As it is explained in the HC-06 datasheet, it expects commands, when it is in AT mode, in an interval of 1 second with 
no newlines and carriage returns required, thus we should send the commands as a string, rather than immediately sending
each character, like \code{minicom} and \code{screen} do by default.

That is why, an easy setup is setting our devices permissions via \code{chmod o+rw /dev/ttyUSB1} and opening 2 terminals:
\begin{enumerate}
    \item One with \code{cat /dev/ttyUSB1}, where we will see all the messages transmitted from the bluetooth module.
    \item One where we can send our commands via \code{echo}, e.g. 
    
        \code{echo 'AT+NAMEABCDE'>/dev/ttyUSB1}.
\end{enumerate}

Another option is to use PuTTY with option \textbf{Local line editing} set to \textbf{Force on}, as it is explained here:
\href{https://stackoverflow.com/questions/4999280/how-to-send-characters-in-putty-serial-communication-only-when-pressing-enter}{
Setting PuTTY to send data serially only on \emph{Enter} key press}.

\section{Issue with AT command not working}
Command \code{AT}, which should make bluetooth module send \code{OK}, doesn't work on my device, although, setting name
with \code{AT+NAMEMY-HC-06} worked and I got \code{OKsetname} back from the bluetooth module. The similar issue is 
described in this thread: \href{https://forum.arduino.cc/t/solved-hc-06-just-wont-talk-with-windows-for-at-commands/665785/8}{
issue with HC-06 \code{AT} command}.

\section{USB bandwidth}
A good article that describes microframes, packet sizes, and intervals: 
\href{https://www.thegoodpenguin.co.uk/blog/understanding-why-usb-isochronous-bandwidth-errors-occur/}{USB bandwidth}.

The used USB device in our FTDI module is USB Mini-B, which is considered a high-speed USB device, just like USB 2.0, as 
it is illustrated on the chart here: \href{https://www.centralcomputer.com/usb-comparison-chart}{USB types and their speeds}.



\end{document}
